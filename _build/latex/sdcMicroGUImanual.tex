%% Generated by Sphinx.
\def\sphinxdocclass{report}
\documentclass[letterpaper,10pt,english]{sphinxmanual}
\ifdefined\pdfpxdimen
   \let\sphinxpxdimen\pdfpxdimen\else\newdimen\sphinxpxdimen
\fi \sphinxpxdimen=.75bp\relax

\PassOptionsToPackage{warn}{textcomp}
\usepackage[utf8]{inputenc}
\ifdefined\DeclareUnicodeCharacter
 \ifdefined\DeclareUnicodeCharacterAsOptional
  \DeclareUnicodeCharacter{"00A0}{\nobreakspace}
  \DeclareUnicodeCharacter{"2500}{\sphinxunichar{2500}}
  \DeclareUnicodeCharacter{"2502}{\sphinxunichar{2502}}
  \DeclareUnicodeCharacter{"2514}{\sphinxunichar{2514}}
  \DeclareUnicodeCharacter{"251C}{\sphinxunichar{251C}}
  \DeclareUnicodeCharacter{"2572}{\textbackslash}
 \else
  \DeclareUnicodeCharacter{00A0}{\nobreakspace}
  \DeclareUnicodeCharacter{2500}{\sphinxunichar{2500}}
  \DeclareUnicodeCharacter{2502}{\sphinxunichar{2502}}
  \DeclareUnicodeCharacter{2514}{\sphinxunichar{2514}}
  \DeclareUnicodeCharacter{251C}{\sphinxunichar{251C}}
  \DeclareUnicodeCharacter{2572}{\textbackslash}
 \fi
\fi
\usepackage{cmap}
\usepackage[T1]{fontenc}
\usepackage{amsmath,amssymb,amstext}
\usepackage{babel}
\usepackage{times}
\usepackage[Bjarne]{fncychap}
\usepackage[,numfigreset=1,mathnumfig]{sphinx}

\usepackage{geometry}

% Include hyperref last.
\usepackage{hyperref}
% Fix anchor placement for figures with captions.
\usepackage{hypcap}% it must be loaded after hyperref.
% Set up styles of URL: it should be placed after hyperref.
\urlstyle{same}
\addto\captionsenglish{\renewcommand{\contentsname}{Table of contents}}

\addto\captionsenglish{\renewcommand{\figurename}{Fig.}}
\addto\captionsenglish{\renewcommand{\tablename}{Table}}
\addto\captionsenglish{\renewcommand{\literalblockname}{Listing}}

\addto\captionsenglish{\renewcommand{\literalblockcontinuedname}{continued from previous page}}
\addto\captionsenglish{\renewcommand{\literalblockcontinuesname}{continues on next page}}

\addto\extrasenglish{\def\pageautorefname{page}}

\setcounter{tocdepth}{1}



\title{sdcMicro GUI manual Documentation}
\date{Jan 04, 2019}
\release{}
\author{Thijs Benschop}
\newcommand{\sphinxlogo}{\vbox{}}
\renewcommand{\releasename}{}
\makeindex

\begin{document}

\maketitle
\sphinxtableofcontents
\phantomsection\label{\detokenize{index::doc}}


This is documentation and guidance for \sphinxstyleemphasis{sdcApp}, a user interface for the \sphinxstyleemphasis{sdcMicro} \sphinxstyleemphasis{R}
package, which provides tools for Statistical Disclosure Control (SDC) for microdata,
also known as microdata anonymization.


\chapter{Introduction}
\label{\detokenize{introduction::doc}}\label{\detokenize{introduction:introduction}}\label{\detokenize{introduction:sdcapp-manual}}

\section{What is sdcApp?}
\label{\detokenize{introduction:what-is-sdcapp}}
\sphinxstyleemphasis{sdcApp} is the Graphical User Interface (GUI) for the R package \sphinxstyleemphasis{sdcMicro} (see
\sphinxhref{https://cran.r-project.org/web/packages/sdcMicro/index.html}{here}). The \sphinxstyleemphasis{sdcApp}
opens the functionality of sdcMicro to users not familiar with the statistical
programming language \sphinxstyleemphasis{R}. \sphinxstyleemphasis{sdcMicro} is an add-on package for the statistical software \sphinxstyleemphasis{R}
for Statistical Disclosure Control (SDC) of microdata and includes functions for risk measurement,
anonymization and utility measurement for
microdata. All functionality available in the \sphinxstyleemphasis{sdcMicro} package is also available in \sphinxstyleemphasis{sdcApp}.


\section{Statistical Disclosure Control (SDC)}
\label{\detokenize{introduction:statistical-disclosure-control-sdc}}
A large part of the data collected by statistical agencies cannot be published directly
due to privacy and confidentiality concerns. These concerns are both of legal and ethical
nature. SDC seeks to treat and alter the data so that the data can be published or
released without revealing the confidential information it contains, while, at the same time,
limit information loss due to the anonymization of the data. There are two strands of literature
on SDC: 1) for microdata and 2) for tabular data. \sphinxstyleemphasis{sdcMicro} and \sphinxstyleemphasis{sdcApp} only provide the tools
and methodology for protecting microdata.


\section{What is the purpose of the manual?}
\label{\detokenize{introduction:what-is-the-purpose-of-the-manual}}
This manual is designed to provide step-by-step guidance through the process of anonymizing a
dataset with microdata. Both the background information on methods and measures is
provided as well as instructions on how to complete these steps in sdcApp. As \sphinxstyleemphasis{sdcApp} is a
GUI for the \sphinxstyleemphasis{sdcMicro} package, users familiar with using \sphinxstyleemphasis{R} for statistical analysis
may prefer to carry out the anonymization process using \sphinxstyleemphasis{R} from command-line.
More information and guidance on using \sphinxstyleemphasis{sdcMicro} from command-line
is available in the SDC Practice Guide available \sphinxhref{https://sdcpractice.readthedocs.io/en/latest/}{here}.
This guide also provides more detailed background information on SDC.


\section{Background literature on SDC for microdata}
\label{\detokenize{introduction:background-literature-on-sdc-for-microdata}}
Add here links to websites/books


\section{Outline of this guide}
\label{\detokenize{introduction:outline-of-this-guide}}
This guide is divided into the following main sections:
\begin{enumerate}
\item {} 
the Section \sphinxhref{installation.html}{Installation and updating} guides the user through the installation process of sdcApp, which includes the installation of R, RStudio as well as the required packages. It also discusses the need and process of regular updates of all software components.

\item {} 
the Section \sphinxhref{introsdcApp.html}{Introduction to sdcApp} covers how to launch and close the application and provides a brief overview of structure of the application.  of the structure of the application

\item {} 
the Section \sphinxhref{loadprepdata.html}{Loading and preparing data} describes how to load microdata into the application. It also discusses the requirements to the

\item {} 
the Section \sphinxhref{setup.html}{Setup anonymization problem}  covers the variable selection and setup of an SDC problem.

\item {} 
the Section \sphinxhref{risk.html}{Risk measurement} covers methods to measure the disclosure risk in the microdata.

\item {} 
the Section \sphinxhref{anon.html}{Anonymization methods} covers anonymization methods for quantitative and qualitative variables.

\item {} 
the Section \sphinxhref{utility.html}{Utility measurement} covers the measurement of information loss resulting from anonymization of the data

\item {} 
the Section \sphinxhref{export.html}{Export data and reports} describes how to export the anonymized dataset and generate reports.

\item {} 
the Section \sphinxhref{reproducibility.html}{Reproducibility} covers functionality that render the anonymization process reproducable.

\end{enumerate}

? Add case study ?


\chapter{Installation and updating}
\label{\detokenize{installation::doc}}\label{\detokenize{installation:installation-and-updating}}
This section will guide you through the steps you need to take to install \sphinxstyleemphasis{sdcApp}.
\sphinxstyleemphasis{sdcApp} is a graphical user interface for the sdcMicro package.
The sdcMicro package is an add-on package for the statistical software R. In order
to start working with \sphinxstyleemphasis{sdcApp}, you need to install R, RStudio %
\begin{footnote}[1]\sphinxAtStartFootnote
Technically speaking, RStudio is not required to run \sphinxstyleemphasis{sdcApp}. Nevertheless, we recommend to install RStudio for a better user experience.
%
\end{footnote} as well as several add-on packages
fpr R. All software is available free of charge and open-source. R and RStudio run on
most platfroms, including Windows, Mac OS X and Linux. To use \sphinxstyleemphasis{sdcApp},
a webbrowser needs to be installed as well.

R, RStudio, the sdcMicro package as well as dependencies are regularly updated. Therefore,
it is recommended to regularly update to the latest version of the software.
The Section {\hyperref[\detokenize{installation:updating-r-rstudio-and-the-sdcmicro-package}]{\sphinxcrossref{Updating R, RStudio and the sdcMicro package}}} shows how to check for updates and install updates.


\section{Installing R and RStudio}
\label{\detokenize{installation:installing-r-and-rstudio}}
The first step in the installation of \sphinxstyleemphasis{sdcApp} is the installation of R and RStudio. The
free open source statistical software R can be downloaded from the \sphinxhref{https://cran.r-project.org}{CRAN website}.
By selecting your OS, the installer will be dowloaded to your computer. In order to install
R, open the installer and follow the installation steps

\begin{figure}[htbp]
\centering
\capstart

\noindent\sphinxincludegraphics{{downloadR}.png}
\caption{Select OS to start downloading R}\label{\detokenize{installation:fig21}}\label{\detokenize{installation:id7}}\end{figure}

Once R is successfully installed, we can install RStudio. RStudio is an IDE (integrated development environment) for R.
RStudio is makes working with R much easier. RStudio Desktop and RStudio Server can be downloaded
free of charge from the \sphinxhref{https://www.rstudio.com/products/rstudio/download/}{RStudio} website.
On this webpage, scroll down to the overview of different versions as shown in Figure  and
select the version corresponding to your OS to start downloading.
In order to install RStudio, open the installer and follow the installation steps.

\begin{figure}[htbp]
\centering
\capstart

\noindent\sphinxincludegraphics{{RStudioVersion}.png}
\caption{Select version to start downloading RStudio}\label{\detokenize{installation:fig22}}\label{\detokenize{installation:id8}}\end{figure}

\begin{sphinxadmonition}{note}{Note:}
We recommend updating to the latest versions of R and RStudio if this software is already
installed on your computer before moving on.
See also the Sections {\hyperref[\detokenize{installation:updating-r}]{\sphinxcrossref{Updating R}}} and {\hyperref[\detokenize{installation:updating-rstudio}]{\sphinxcrossref{Updating RStudio}}} for more information on updating the software.
\end{sphinxadmonition}

Once R and RStudio are installed on your computer, open RStudio. The RStudio interface consists
of four different panes as shown in Figure.
\begin{enumerate}
\item {} 
script editor (by default left up)

\item {} 
(by default right up)

\item {} 
R console (by default left down)

\item {} 
(by default right down)

\end{enumerate}

\begin{figure}[htbp]
\centering
\capstart

\noindent\sphinxincludegraphics{{RStudio}.png}
\caption{Screenshot RStudio}\label{\detokenize{installation:fig23}}\label{\detokenize{installation:id9}}\end{figure}


\section{sdcMicro package}
\label{\detokenize{installation:sdcmicro-package}}
\sphinxstyleemphasis{sdcApp} is included in the R package \sphinxstyleemphasis{sdcMicro}. Once RStudio is open, \sphinxstyleemphasis{sdcMicro} can be
installed by executing commands in the R console. \sphinxstyleemphasis{sdcMicro} and a set of other R packages
that are required by the \sphinxstyleemphasis{sdcMicro} package are downloaded from the CRAN servers. Therefore,
it is necessary to be connected to the internet during the installation process. %
\begin{footnote}[2]\sphinxAtStartFootnote
It is possible to download R, RStudio and the packages and transfer the files to the computer with for example a USB drive in case the computer
sdcMicro should be installed on cannot be connected to the internet for technical or confidentiality reasons.
%
\end{footnote}

\begin{figure}[htbp]
\centering
\capstart

\noindent\sphinxincludegraphics{{downloadR}.png}
\caption{Screenshot RStudio}\label{\detokenize{installation:fig24}}\label{\detokenize{installation:id10}}\end{figure}

In order to install the latest version of the \sphinxstyleemphasis{sdcMicro} package, type the command
\sphinxcode{\sphinxupquote{install.packages("sdcMicro", dependencies = TRUE)}} in the console and press enter to execute.
The first time you are installing R packages, a prompt will ask you to select a CRAN mirror (server) to install the package from.
Since the packages on all mirrors are identical, you can choose any of the locations.
The sdcMicro package itself uses functionality
from a set of other R packages (e.g., \sphinxstyleemphasis{haven} for reading files in different formats).
By specifying the dependencies argument to TRUE, these dependencies will automatically be installed too.

\def\sphinxLiteralBlockLabel{\label{\detokenize{installation:code01}}}
\sphinxSetupCaptionForVerbatim{Installing sdcMicro package}
\fvset{hllines={, ,}}%
\begin{sphinxVerbatim}[commandchars=\\\{\},numbers=left,firstnumber=1,stepnumber=1]
\PYG{c+c1}{\PYGZsh{} install sdcMicro package}
install.packages\PYG{p}{(}\PYG{l+s}{\PYGZdq{}}\PYG{l+s}{sdcMicro\PYGZdq{}}\PYG{p}{,} dependencies \PYG{o}{=} \PYG{k+kc}{TRUE}\PYG{p}{)}
\end{sphinxVerbatim}

\begin{sphinxadmonition}{note}{Note:}
Also dependencies will be installed and the installation may take some time.
Dependencies are other add-on packages, of which the functionality is required to run the sdcMicro package.
\end{sphinxadmonition}

\begin{sphinxadmonition}{note}{Note:}
An internet connection is not required while using \sphinxstyleemphasis{sdcMicro} and \sphinxstyleemphasis{sdcApp} and the data
are stored locally on your computer or server. The web browser uses a local host IP,
which is not connected to the internet and the browser is only used to communicate with
the running R session.
\end{sphinxadmonition}


\section{Launching \sphinxstyleemphasis{sdcApp}}
\label{\detokenize{installation:launching-sdcapp}}
Once the sdcMicro package is successfully installed, the sdcMicro package needs to be loaded.
Installing the package is only required once (except for updating), whereas loading the
package is required every time a new R session is started.

You can load the sdcMicro package by typing library(sdcMicro) and launch the application by typing \sphinxcode{\sphinxupquote{sdcApp()}}.

\sphinxstyleemphasis{sdcApp} opens in your system’s default web browser through the local host IP \sphinxcode{\sphinxupquote{127.0.0.1:}}.
\sphinxstyleemphasis{sdcApp} works with recent versions of any webbrowser.
Due to small issues encountered with some browsers, we recommend to use Google Chrome, Mozilla Firefox or Safari for the best performance.
In case your default web browser is not one of the aforementioned browsers, you can simply open an
alternative browser and copy paste the internal IP address to the new browser. \sphinxstyleemphasis{sdcApp} will open
in the new browser.

Furthermore, it’s important that your R session is allowed to use the installed webbrowser.

\begin{figure}[htbp]
\centering
\capstart

\noindent\sphinxincludegraphics{{RconsoleIP}.png}
\caption{R console with local IP after launching \sphinxstyleemphasis{sdcApp}}\label{\detokenize{installation:fig25}}\label{\detokenize{installation:id11}}\end{figure}

\begin{figure}[htbp]
\centering
\capstart

\noindent\sphinxincludegraphics{{sdcAppStartIP}.png}
\caption{Start screen sdcApp in browser with local IP}\label{\detokenize{installation:fig26}}\label{\detokenize{installation:id12}}\end{figure}

\def\sphinxLiteralBlockLabel{\label{\detokenize{installation:code02}}}
\sphinxSetupCaptionForVerbatim{Loading sdcMicro package and launching \sphinxstyleemphasis{sdcApp}}
\fvset{hllines={, ,}}%
\begin{sphinxVerbatim}[commandchars=\\\{\},numbers=left,firstnumber=1,stepnumber=1]
\PYG{c+c1}{\PYGZsh{} Load sdcMicro package}
\PYG{k+kn}{library}\PYG{p}{(}sdcMicro\PYG{p}{)}

\PYG{c+c1}{\PYGZsh{} Launch sdcApp (opens in browser window)}
sdcApp\PYG{p}{(}\PYG{p}{)}
\end{sphinxVerbatim}

In rare cases, not all dependencies are correctly installed and the following error
message appears in the R console upon loading the sdcMicro package.

Please install the package(s) indicated in the error message manually by using the
command install.packages() with the name of the package(s). In the example error message,
this would be for the packages .


\section{Updating R, RStudio and the sdcMicro package}
\label{\detokenize{installation:updating-r-rstudio-and-the-sdcmicro-package}}
R, RStudio, the sdcMicro package as well as dependencies are regularly updated. Updates include
bug fixes as well as additional functionality. Therefore,
it is recommended to regularly update to the latest version of the software.


\subsection{Updating R}
\label{\detokenize{installation:updating-r}}
RStudio uses by default the most recent version of R available on your system. New
versions of R packages, including the \sphinxstyleemphasis{sdcMicro} package, rely on the newest version of R. Therefore,
it’s important to regularly check for updates of R. The easiest way to do so
is to visit regularly the \sphinxhref{https://cran.r-project.org}{CRAN website}.
If a new version of R is available, the same steps as as for the installation of R need to be followed
as described in the Section {\hyperref[\detokenize{installation:installing-r-and-rstudio}]{\sphinxcrossref{Installing R and RStudio}}}. The version number of the
R version installed on your computer appears in the R console upon launching R or RStudio
(cf. Figure ).

\begin{figure}[htbp]
\centering
\capstart

\noindent\sphinxincludegraphics{{Rversion}.png}
\caption{R console with version number}\label{\detokenize{installation:fig27}}\label{\detokenize{installation:id13}}\end{figure}


\subsection{Updating RStudio}
\label{\detokenize{installation:updating-rstudio}}
RStudio has Help -\textgreater{} Check for updates. If an update is available, the current version number and the
newest version number are shown. In order to install the newer version, you need to visit the
\sphinxhref{https://www.rstudio.com/products/rstudio/download/}{RStudio} website and follow the steps
as described in the Section {\hyperref[\detokenize{installation:installing-r-and-rstudio}]{\sphinxcrossref{Installing R and RStudio}}}.


\subsection{Updating R packages}
\label{\detokenize{installation:updating-r-packages}}
The sdcMicro package is regularly updated to fix bugs and add functionality. In order to check
for newer versions, click on the Update button to get an overview of all packages that have
newer versions available. By clicking Select all, these packages are all automatically updated.

Report a bug

\def\sphinxLiteralBlockLabel{\label{\detokenize{installation:code03}}}
\sphinxSetupCaptionForVerbatim{Updating packages}
\fvset{hllines={, ,}}%
\begin{sphinxVerbatim}[commandchars=\\\{\},numbers=left,firstnumber=1,stepnumber=1]
\PYG{c+c1}{\PYGZsh{} Update}
install.packages\PYG{p}{(}\PYG{p}{)}
\end{sphinxVerbatim}


\section{Bug reporting on GitHub}
\label{\detokenize{installation:bug-reporting-on-github}}
The sdcMicro package is open source software and the source code can be easily viewed on
the \sphinxhref{https://github.com/sdcTools/sdcMicro}{GitHub} of the sdcMicro project. There you can
also report alleged bugs and raise other issues.


\chapter{Introduction to sdcApp}
\label{\detokenize{introsdcApp::doc}}\label{\detokenize{introsdcApp:introduction-to-sdcapp}}
\sphinxstyleemphasis{sdcApp} is a user-friendly application for microdata anonymization
and is built on the \sphinxhref{https://shiny.rstudio.com}{Shiny}
technology. Shiny allows users to
communicate through a GUI that runs in a webbrowser with a local \sphinxstyleemphasis{R} session. The local
\sphinxstyleemphasis{R} session performs the necessary calculations. In the case of \sphinxstyleemphasis{sdcApp}, most functionality
used in \sphinxstyleemphasis{R} is included in the \sphinxhref{https://CRAN.R-project.org/package=sdcMicro}{sdcMicro}
package.

\sphinxstyleemphasis{sdcApp} has a tab structure and consists of seven tabs, which in turn consist of
up to three panels. This structured is further explored below.
The tabs and panels are used to navigate through the app.


\section{Starting sdcApp}
\label{\detokenize{introsdcApp:starting-sdcapp}}
After succcesful installation (see the Section \sphinxhref{installation.html}{Installation and updating}),
\sphinxstyleemphasis{sdcApp} is ready for use. Every single time \sphinxstyleemphasis{sdcApp} is used,
first the applications \sphinxstyleemphasis{R} or \sphinxstyleemphasis{RStudio} need to be opened. We recommend to use \sphinxstyleemphasis{RStudio}
for ease of use. After launching \sphinxstyleemphasis{R} or \sphinxstyleemphasis{RStudio},
the \sphinxstyleemphasis{sdcMicro} package needs to be loaded and \sphinxstyleemphasis{sdcApp} needs to be launched.
To load \sphinxstyleemphasis{sdcMicro} and launch \sphinxstyleemphasis{sdcApp}, enter the code as shown in \hyperref[\detokenize{introsdcApp:code1}]{Listing \ref{\detokenize{introsdcApp:code1}}} in the \sphinxstyleemphasis{R} console.
Press enter after each line to execute the line of code. \hyperref[\detokenize{introsdcApp:fig31}]{Fig.\@ \ref{\detokenize{introsdcApp:fig31}}} shows the
output in the \sphinxstyleemphasis{R} console after successfully launching \sphinxstyleemphasis{sdcApp}.

\def\sphinxLiteralBlockLabel{\label{\detokenize{introsdcApp:code1}}}
\sphinxSetupCaptionForVerbatim{Loading sdcMicro package and launching sdcApp}
\fvset{hllines={, ,}}%
\begin{sphinxVerbatim}[commandchars=\\\{\},numbers=left,firstnumber=1,stepnumber=1]
\PYG{c+c1}{\PYGZsh{} Load sdcMicro package}
\PYG{k+kn}{library}\PYG{p}{(}sdcMicro\PYG{p}{)}

\PYG{c+c1}{\PYGZsh{} Launch sdcApp (opens in browser window)}
sdcApp\PYG{p}{(}\PYG{p}{)}
\end{sphinxVerbatim}

\begin{sphinxadmonition}{note}{Note:}
You can omit the lines starting with a hash tag (\sphinxcode{\sphinxupquote{\#}}) as these are comment lines
and ignored by the \sphinxstyleemphasis{R} interpreter.
\end{sphinxadmonition}

The application opens in a new tab in your default web browser. In case you prefer to use
an alternative web browser, you can simply copy the address of the localhost and paste it into
a different browser on the same machine. The localhost address can be
found in the output in the \sphinxstyleemphasis{R} console. The address starts with \sphinxcode{\sphinxupquote{http://127.0.0.1:}} followed
by a four digit number. In the example in \hyperref[\detokenize{introsdcApp:fig31}]{Fig.\@ \ref{\detokenize{introsdcApp:fig31}}}, the full localhost address is
\sphinxcode{\sphinxupquote{http://127.0.0.1:3256}}. The application opens on the \sphinxstylestrong{About/Help} tab (see \hyperref[\detokenize{introsdcApp:fig32}]{Fig.\@ \ref{\detokenize{introsdcApp:fig32}}}).

\begin{sphinxadmonition}{note}{Note:}
Firewalls and other settings on your computer and browser may prevent \sphinxstyleemphasis{sdcApp} from opening
in your webbrowser. As a first thing you could try to copy paste the localhost address
into your webbrowser. If that is not successful, try changing the settings of your
browser and firewall.
\end{sphinxadmonition}

\begin{figure}[htbp]
\centering
\capstart

\noindent\sphinxincludegraphics{{introRConsoleLaunch}.png}
\caption{R console after loading the \sphinxstyleemphasis{sdcMicro} package and launching \sphinxstyleemphasis{sdcApp}}\label{\detokenize{introsdcApp:fig31}}\label{\detokenize{introsdcApp:id1}}\end{figure}

\begin{figure}[htbp]
\centering
\capstart

\noindent\sphinxincludegraphics{{introLandingPage}.png}
\caption{\sphinxstylestrong{About/Help} tab in web browser after launching \sphinxstyleemphasis{sdcApp} with localhost address}\label{\detokenize{introsdcApp:fig32}}\label{\detokenize{introsdcApp:id2}}\end{figure}


\section{Tab and panel structure}
\label{\detokenize{introsdcApp:tab-and-panel-structure}}
The sdcApp consists of seven different tabs that serve different parts of the
SDC process. The tabs can be selected in the navigation
bar at the top of the page (cf. the area indicated with 1 in \hyperref[\detokenize{introsdcApp:fig33}]{Fig.\@ \ref{\detokenize{introsdcApp:fig33}}}).
The navigation bar is visible at all times. The content of each tab may change as
function the specified SDC problem and the current state of the SDC process. For example,
anonymization methods for continuous key variables are not shown on the \sphinxstylestrong{Anonymize} tab,
if no continuous key variables are selected.
\begin{itemize}
\item {} \begin{description}
\item[{About/Help}] \leavevmode
Landing page to set storage path, quit and restart \sphinxstyleemphasis{sdcApp} as well as provide feedback to the developers

\end{description}

\item {} \begin{description}
\item[{Microdata}] \leavevmode
Page to load, view, explore and prepare the microdata to be anonymized

\end{description}

\item {} \begin{description}
\item[{Anonymize}] \leavevmode
Page to setup the anonymization problem (select variables, set parameters). Once the
problem is defined, this page shows a summary of the anonymization problem and allows
to apply anonymization methods

\end{description}

\item {} \begin{description}
\item[{Risk/Utility}] \leavevmode
Page to evaluate disclosure risk and information loss (data utility)

\end{description}

\item {} \begin{description}
\item[{Export Data}] \leavevmode
Page to export the anonymized data and reports on the anonymization process

\end{description}

\item {} \begin{description}
\item[{Reproducibility}] \leavevmode
Page with functionality to guarantee the reproducibility of the process by exporting the
\sphinxstyleemphasis{R} script or problem instance

\end{description}

\item {} \begin{description}
\item[{Undo}] \leavevmode
Page to revert one or several steps in the anonymization process

\end{description}

\end{itemize}

Each tab consists of two panels: the left sidebar (cf. the area indicated with 2 in \hyperref[\detokenize{introsdcApp:fig33}]{Fig.\@ \ref{\detokenize{introsdcApp:fig33}}})
and the main panel (cf. the area indicated with 3 in \hyperref[\detokenize{introsdcApp:fig33}]{Fig.\@ \ref{\detokenize{introsdcApp:fig33}}}). The left panel
allows the user to navigate between different function on the same tab, e.g., different
risk measures. Some tabs have an additional right sidebar (cf. the area indicated with 4 in \hyperref[\detokenize{introsdcApp:fig33}]{Fig.\@ \ref{\detokenize{introsdcApp:fig33}}}),
which provide summary information on the current SDC problem.

\begin{figure}[htbp]
\centering
\capstart

\noindent\sphinxincludegraphics{{appStructure}.png}
\caption{Risk/Utility tab with navigation bar and panel structure}\label{\detokenize{introsdcApp:fig33}}\label{\detokenize{introsdcApp:id3}}\end{figure}


\section{In-app help}
\label{\detokenize{introsdcApp:in-app-help}}
By hovering over the \sphinxstyleemphasis{i} icon in \sphinxstyleemphasis{sdcApp}, additional information on e.g.,
specific parameters and the interpretation of results is provided. The help information
is mainly intended to provide a brief reminder and is not meant to replace a thorough
study of the SDC literature on risk and utility measurement and anonymization methods.
\hyperref[\detokenize{introsdcApp:fig36}]{Fig.\@ \ref{\detokenize{introsdcApp:fig36}}} shows the help pop-up for the variable selection table.

\begin{figure}[htbp]
\centering
\capstart

\noindent\sphinxincludegraphics{{introHelp}.png}
\caption{Help pop-up when moving with mouse cursor over \sphinxstyleemphasis{i} icon}\label{\detokenize{introsdcApp:fig36}}\label{\detokenize{introsdcApp:id4}}\end{figure}


\section{Getting started}
\label{\detokenize{introsdcApp:getting-started}}
Use testdata dataset: all examples in this guide are illustrated with the testdata dataset.


\section{Set storage path}
\label{\detokenize{introsdcApp:set-storage-path}}
All output exported from \sphinxstyleemphasis{sdcApp}, such as the anonymized dataset, reports and scripts will be
saved in the directory shown under the header \sphinxstylestrong{Set storage path} on the \sphinxstylestrong{About/Help}
tab (cf. \hyperref[\detokenize{introsdcApp:fig34}]{Fig.\@ \ref{\detokenize{introsdcApp:fig34}}}). Upon launching \sphinxstyleemphasis{sdcApp}, this directory is set to the \sphinxstyleemphasis{R} working
directory. Change the working directory to a the folder in the project directory with the
dataset to be anonymized by typing the path to this folder in the input box (cf. \hyperref[\detokenize{introsdcApp:fig34}]{Fig.\@ \ref{\detokenize{introsdcApp:fig34}}}).
Once a valid path on your computer is entered, click the blue button \sphinxstylestrong{Update the current output
path} to change the path. If the entered path is not a valid path on your system, a red button appears
with the text \sphinxstylestrong{The specified directory does not exist, thus the path can’t be updated}.
It is recommended to create a new folder in the project directory for the \sphinxstyleemphasis{sdcApp} output.
The file names of the output files contain a date and time stamp as well as a brief description,
e.g., exportedData\_sdcMicro\_20181010\_1211.dta for the anonymized microdata in STATA format
on October 10, 2018 at 12:11 and exportedProblem\_sdcMicro\_20180304\_1633.rdata for the
saved problem instance as \sphinxstyleemphasis{R} datafile on March 4, 2018 ar 16:33.

\begin{sphinxadmonition}{note}{Note:}
The storage path to the output folder needs to be specified every time \sphinxstyleemphasis{sdcApp} is
launched.
\end{sphinxadmonition}

\begin{sphinxadmonition}{note}{Note:}
If an sdcProblem is saved and reloaded, the storage path is set to the path saved
in the sdcProblem. If the problem is loaded on a different computer than it was saved at,
the storage path may be invalid and needs to be updated in the same way as described above.
\end{sphinxadmonition}

\begin{figure}[htbp]
\centering
\capstart

\noindent\sphinxincludegraphics{{introSetStoragePath}.png}
\caption{View and set storage path for file export}\label{\detokenize{introsdcApp:fig34}}\label{\detokenize{introsdcApp:id5}}\end{figure}


\section{Quiting sdcApp}
\label{\detokenize{introsdcApp:quiting-sdcapp}}
To quit \sphinxstyleemphasis{sdcApp}, click on \sphinxstylestrong{Stop the GUI} under the header \sphinxstylestrong{Stop the interface} on the
\sphinxstylestrong{About/Help} tab. It is recommended to quit \sphinxstyleemphasis{R} or \sphinxstyleemphasis{RStudio} after quitting \sphinxstyleemphasis{sdcApp}
to ensure that nothing is left in the memory. This especially applies to a restart due
to \sphinxstyleemphasis{sdcApp} not responding.

\begin{figure}[htbp]
\centering
\capstart

\noindent\sphinxincludegraphics{{introStopGUI}.png}
\caption{Button to quit \sphinxstyleemphasis{sdcApp} on the \sphinxstylestrong{About/Help} tab}\label{\detokenize{introsdcApp:fig35}}\label{\detokenize{introsdcApp:id6}}\end{figure}

Save SDC problem to continue working later. Possible once the SDC problem is defined.
See undo section


\chapter{Loading and Preparing Data}
\label{\detokenize{loadprepdata::doc}}\label{\detokenize{loadprepdata:loading-and-preparing-data}}
This section discusses how to load microdata into \sphinxstyleemphasis{sdcApp} and prepare the data
for the SDC process.

The first step in the SDC process is loading the dataset into \sphinxstyleemphasis{sdcApp}. \sphinxstyleemphasis{sdcApp} supports
most common data formats, such as \sphinxstyleemphasis{R}, \sphinxstyleemphasis{STATA}, \sphinxstyleemphasis{SPSS} and \sphinxstyleemphasis{SAS} files. First time users
may also load one of the two practice datasets, which are included in \sphinxstyleemphasis{sdcApp},
to explore \sphinxstyleemphasis{sdcApp} and methods. Most examples in this guide are illustrated by
using the practice dataset \sphinxstyleemphasis{testdata} and can be reproduced.

After loading the data, the user needs to prepare the data for the SDC process.
Most preparation steps can be carried out in \sphinxstyleemphasis{sdcApp}, although users may find it
more convenient to perform some of these actions in another statistical software
before loading the data in \sphinxstyleemphasis{sdcApp}.


\section{Loading data}
\label{\detokenize{loadprepdata:loading-data}}

\subsection{Testdata}
\label{\detokenize{loadprepdata:testdata}}
\sphinxstyleemphasis{sdcApp} includes two practice datasets: \sphinxstyleemphasis{testdata} and \sphinxstyleemphasis{testdata2}. The dataset
\sphinxstyleemphasis{testdata} is used to illustrate methods and examples in this guide. In order to
replicate these examples, the user needs to load this dataset. In order to load the testdata
dataset, navigate to the \sphinxstylestrong{Microdata tab} and select \sphinxstylestrong{Testdata/Internal data} in the left sidebar.
Select the dataset from the dropdown menu and click the button \sphinxstylestrong{Load data}.
This is illustrated in \hyperref[\detokenize{loadprepdata:fig51}]{Fig.\@ \ref{\detokenize{loadprepdata:fig51}}}. After loading the testdata dataset, the loaded
dataset is displayed (cf. \hyperref[\detokenize{loadprepdata:fig52}]{Fig.\@ \ref{\detokenize{loadprepdata:fig52}}}).

Any other datasets loaded in the current \sphinxstyleemphasis{R} session are also shown in the list with
available datasets and can be loaded.

\begin{figure}[htbp]
\centering
\capstart

\noindent\sphinxincludegraphics{{prepLoadTestdata}.png}
\caption{Load testdata on \sphinxstylestrong{Microdata} tab}\label{\detokenize{loadprepdata:fig51}}\label{\detokenize{loadprepdata:id1}}\end{figure}

\begin{figure}[htbp]
\centering
\capstart

\noindent\sphinxincludegraphics{{prepLoadedData}.png}
\caption{Loaded dataset}\label{\detokenize{loadprepdata:fig52}}\label{\detokenize{loadprepdata:id2}}\end{figure}


\subsection{Other microdata}
\label{\detokenize{loadprepdata:other-microdata}}
\sphinxstyleemphasis{sdcApp} supports datasets in several foreign data formats (cf. \hyperref[\detokenize{loadprepdata:tab51}]{Table \ref{\detokenize{loadprepdata:tab51}}}).
If the microdata is not in one of these data formats, another software can be used
to convert the data, such as \sphinxstyleemphasis{Stat/Transfer}. Also some statistical software allow to export
the data in another data format.


\begin{savenotes}\sphinxattablestart
\centering
\sphinxcapstartof{table}
\sphinxcaption{Data formats compatible with sdcApp}\label{\detokenize{loadprepdata:tab51}}\label{\detokenize{loadprepdata:id3}}
\sphinxaftercaption
\begin{tabulary}{\linewidth}[t]{|T|T|}
\hline
\sphinxstyletheadfamily 
Software
&\sphinxstyletheadfamily 
File extension
\\
\hline
R/RStudio
&
.rdata
\\
\hline
SPSS
&
.sav
\\
\hline
SAS
&
.sas7bdat
\\
\hline
CSV
&
.csv, .txt
\\
\hline
STATA
&
.dta
\\
\hline
\end{tabulary}
\par
\sphinxattableend\end{savenotes}

In order to load a dataset, select the corresponding data format
from the left sidebar of the Microdata tab. For all formats the user can set two options:
\begin{enumerate}
\item {} \begin{description}
\item[{Convert string variables (character vectors) to factor variables?}] \leavevmode
If \sphinxcode{\sphinxupquote{TRUE}} (default), variables of type string are automatically converted to categorical variables
(type factor in \sphinxstyleemphasis{R}). Categorical variables need to be of type factor in \sphinxstyleemphasis{sdcApp}.
Remove any textual variables, such as ‘Specify other:’ variables before loading the
data. These variables are oftentimes not suitable for release and require long
computation times to be transformed to factor. IF \sphinxcode{\sphinxupquote{FALSE}}

\end{description}

\item {} \begin{description}
\item[{Drop variables with only missing values (NA)?}] \leavevmode
If \sphinxcode{\sphinxupquote{TRUE}} (default), variables that contain only missing values (\sphinxcode{\sphinxupquote{NA}} in \sphinxstyleemphasis{R})
are removed upon loading the data. This does not cause any loss of information,
as these variabels do not contain information. However, variables with only
missing values can cause issues in \sphinxstyleemphasis{sdcApp}. If \sphinxcode{\sphinxupquote{FALSE}}, no variables are deleted.

\end{description}

\end{enumerate}

If the selected data format is a CSV-file, two additional options need to be specified:
\begin{enumerate}
\item {} \begin{description}
\item[{Does the first row contain the variable names?}] \leavevmode
If \sphinxcode{\sphinxupquote{TRUE}}, the values in the first row are used as variable names. If
\sphinxcode{\sphinxupquote{FALSE}}, the variables names are set to V1, V2, V3, … in the order of
appearance in the dataset.

\end{description}

\item {} \begin{description}
\item[{Field separator}] \leavevmode
The field separator in the csv file needs to be specified. Options are comma (,),
semicolon (;) and tab.

\end{description}

\end{enumerate}

After setting the options for the data upload, click on the button \sphinxstylestrong{Browse} to access
the file system in your computer and select the microdata file. The file is upload
immediately after selection. After loading the file, which may

\begin{sphinxadmonition}{note}{Note:}
Set the additional options before selecting the datafile from your file system.
Upon selection after clicking \sphinxstylestrong{Browse}, the file is immediately loaded and settings
can no longer be changed. If the file was accidentally loaded before setting all
parameters, the file needs to be reloaded after first restting the microdata by
clicking \sphinxstylestrong{Reset microdata} in the left sidebar.
\end{sphinxadmonition}

\begin{sphinxadmonition}{note}{Note:}
The default maximum file size in \sphinxstyleemphasis{sdcApp} is 50 MB. In order to upload larger files,
the maximum file size in MB needs to be specified upon launching \sphinxstyleemphasis{sdcApp}. This can
be achieved by specifying the argument \sphinxcode{\sphinxupquote{maxRequestSize}}:

\def\sphinxLiteralBlockLabel{\label{\detokenize{loadprepdata:id4}}}
\sphinxSetupCaptionForVerbatim{Launching \sphinxstyleemphasis{sdcApp} to load larger files}
\fvset{hllines={, ,}}%
\begin{sphinxVerbatim}[commandchars=\\\{\},numbers=left,firstnumber=1,stepnumber=1]
\PYG{c+c1}{\PYGZsh{} Launch sdcApp with increased max. file size (200MB)}
sdcApp\PYG{p}{(}maxRequestSize \PYG{o}{=} \PYG{l+m}{200}\PYG{p}{)}
\end{sphinxVerbatim}
\end{sphinxadmonition}

\begin{figure}[htbp]
\centering
\capstart

\noindent\sphinxincludegraphics{{prepLoadData}.png}
\caption{Load data on Microdata tab - example STATA dataset}\label{\detokenize{loadprepdata:fig53}}\label{\detokenize{loadprepdata:id5}}\end{figure}

After loading the dataset, the data is shown in the \sphinxstylestrong{Microdata} tab. The \sphinxstylestrong{Microdata}
tab changes and the functionality for loading microdata is replaced with
functionality to explore and prepare the dataset (cf. \hyperref[\detokenize{loadprepdata:fig511}]{Fig.\@ \ref{\detokenize{loadprepdata:fig511}}}). The
left sidebar shows different options to explore and prepare the data for the anonymization process,
as discussed in the next sections.

\begin{figure}[htbp]
\centering
\capstart

\noindent\sphinxincludegraphics{{prepLoadAfterLoad}.png}
\caption{Microdata tab after loading dataset}\label{\detokenize{loadprepdata:fig511}}\label{\detokenize{loadprepdata:id6}}\end{figure}


\section{Inspect and explore data}
\label{\detokenize{loadprepdata:inspect-and-explore-data}}
After loading the dataset into \sphinxstyleemphasis{sdcApp}, the data is shown on the Microdata tab. At the top of
the data viewer, the number of observations and variables is shown as well as the number
of variables that were deleted as a result all missing values (cf. \hyperref[\detokenize{loadprepdata:fig511}]{Fig.\@ \ref{\detokenize{loadprepdata:fig511}}}).

\begin{sphinxadmonition}{note}{Note:}
If \sphinxstyleemphasis{Drop variables with only missing values (NA)?} is set to TRUE, the number of variables
shown may be lower than the number of variables in the loaded dataset.
\end{sphinxadmonition}

It is important to check whether the data was imported completely and correctly by browsing
the dataset in \sphinxstyleemphasis{sdcApp}. If, for example, records are missing or labels are corrupted,
then these issues need to be fixed outside of \sphinxstyleemphasis{sdcApp} and the data need to be reimported.

By clicking \sphinxstylestrong{Explore variables} in the left sidebar, univariate and bivariate summary
statistics appropriate for the variable type can displayed. If one variable is selected,
univariate summary statistics are shown.

\begin{sphinxadmonition}{note}{Note:}
The choice of summary statistics is based on the variable type specified in \sphinxstyleemphasis{R} (shown in
brackets after the variable name, e.g., urbrur (integer)). Therefore,
the representation may not be correct, if the variable type does not correspond
with the variable content. By converting the variable (see {\hyperref[\detokenize{loadprepdata:convert-variable-type}]{\sphinxcrossref{Convert variable type}}}),
the correct summary statistics will be displayed.
\end{sphinxadmonition}


\section{Preparing data}
\label{\detokenize{loadprepdata:preparing-data}}
Most datasets need to be prepared before the start of the anonymization process. Examples
of data preparation are removing variables that are not suitable for release, etc. It is
recommended to carry out the data preparation in a statistical software of choice, before
loading the data in sdcApp. Data preparation includes

After loading the data in sdcApp, still some steps may need to be carried, which are
specific to the needs of the sdcApp. These steps are discussed in the following subsections.


\subsection{Convert variable type}
\label{\detokenize{loadprepdata:convert-variable-type}}
numeric to factor

to numeric


\subsection{Set specific values to NA}
\label{\detokenize{loadprepdata:set-specific-values-to-na}}
Missing values play an important role in anonymization of microdata. In particular when
measuring disclosure risk of categorical key variables (see {\color{red}\bfseries{}{}`Risk{}`\_\_}). sdcApp only considers
the R missing value \sphinxcode{\sphinxupquote{NA}} as missing. Therefore, it is important to recode other missing values,
such as 9, 99, 998 or 999, “Missing”, “Not applicable” after loading the
data to the R missing value \sphinxcode{\sphinxupquote{NA}}, if appropriate. Many standard missing value codes
in the data, such as \sphinxcode{\sphinxupquote{.}} in STATA are automatically converted to NA upon loading
the data into \sphinxstyleemphasis{sdcApp}.

\begin{figure}[htbp]
\centering
\capstart

\noindent\sphinxincludegraphics{{prepareMissingToNA}.png}
\caption{Screen to set specific value in a variable to NA}\label{\detokenize{loadprepdata:fig57}}\label{\detokenize{loadprepdata:id7}}\end{figure}


\subsection{Modify factor variable}
\label{\detokenize{loadprepdata:modify-factor-variable}}
Recoding (see Recoding)


\subsection{Create stratification variable}
\label{\detokenize{loadprepdata:create-stratification-variable}}
\begin{figure}[htbp]
\centering
\capstart

\noindent\sphinxincludegraphics{{prepStrataVariable}.png}
\caption{Screen to create new stratification variable}\label{\detokenize{loadprepdata:fig58}}\label{\detokenize{loadprepdata:id8}}\end{figure}


\subsection{Reset variables}
\label{\detokenize{loadprepdata:reset-variables}}

\subsection{Hierarchical data}
\label{\detokenize{loadprepdata:hierarchical-data}}
\begin{figure}[htbp]
\centering
\capstart

\noindent\sphinxincludegraphics{{prepHierarchical1}.png}
\caption{Screen to create household level dataset}\label{\detokenize{loadprepdata:fig59}}\label{\detokenize{loadprepdata:id9}}\end{figure}

\begin{figure}[htbp]
\centering
\capstart

\noindent\sphinxincludegraphics{{prepHierarchical2}.png}
\caption{Screen merge anonymized household level dataset with individual level dataset}\label{\detokenize{loadprepdata:fig510}}\label{\detokenize{loadprepdata:id10}}\end{figure}


\subsection{Use subset of microdata}
\label{\detokenize{loadprepdata:use-subset-of-microdata}}

\chapter{Setup anonymization problem}
\label{\detokenize{setup::doc}}\label{\detokenize{setup:setup-anonymization-problem}}
Based on the analysis of the disclosure scenarios (see ), the user needs can make the variable
selection in \sphinxstyleemphasis{sdcApp} and set some other parameters in order to define the
so-called SDC problem. Once the data is loaded and prepared,
the tab \sphinxstyleemphasis{Microdata} shows a variable selection matrix in the main panel. The right sidebar
shows several parameter settings and allows to have a quck summary view of each of the variables
in the loaded dataset.


\section{Variable selection}
\label{\detokenize{setup:variable-selection}}
In order to setup an SDC problem the user needs to make a variable selection. The variable
selection itself is the result of the analysis of diclosure scenarios and is beyond the scope
of this manual. We refer to Chapter in for a thorough discussion of disclosure scenarios.

The matrix shown in \hyperref[\detokenize{setup:fig11}]{Fig.\@ \ref{\detokenize{setup:fig11}}} contain one row for each variable in the loaded dataset
and nine different columns as described in \hyperref[\detokenize{setup:tabsetup1}]{Table \ref{\detokenize{setup:tabsetup1}}}. The user can select for each
variable the function it has in the sDC problem. No selection needs to
be made for variables that are not relevant to the
anonymization process and can be released without further treatment. Each of the different
columns is described in more detail:
\begin{enumerate}
\item {} \begin{description}
\item[{\sphinxstylestrong{Variable name}}] \leavevmode
This column specifies the variable name as provided in the original dataset.
Variable names cannot be changed in \sphinxstyleemphasis{sdcApp}, as they are unique identifiers. If
the anonymization process renders a variable name no longer appropriate, the variable
must be renamed after exporting the dataset in a software of choice.

\end{description}

\item {} \begin{description}
\item[{\sphinxstylestrong{Type}}] \leavevmode
Each variable has a internal \sphinxstyleemphasis{R} type. The different types include
numeric, integer, factor and string. Each of the different functions in the
SDC process requires a specific variable type, e.g., the weight needs to be numeric.
If a variable is not of the appropriate type, the type of the variable needs to be changed
before a selection is made (see the Section \sphinxhref{loadprepdata.html}{Convert variable type}).

\end{description}

\item {} \begin{description}
\item[{\sphinxstylestrong{Key variables}}] \leavevmode
Variables that are determined as key variables in the disclosure
scenario need to be selected here. By default the radiobutton is at \sphinxstyleemphasis{No}. A variable
can either be a categorical key variable (\sphinxstyleemphasis{cat.}) or a numeric key variable (\sphinxstyleemphasis{cont.}).
The sets of categorical key variables and numeric key variables are treated independently
in \sphinxstyleemphasis{sdcApp}. Categorical key variables can be of type integer or factor. Numeric key variables
can be of type integer or numeric. At least one variable needs to be selected as
categorcial key variable in order to create an SDC problem.

\end{description}

\item {} \begin{description}
\item[{\sphinxstylestrong{Weight}}] \leavevmode
The sampling weight is used to measure the disclosure risk. The weight
variable needs to be of type numeric.

\end{description}

\item {} \begin{description}
\item[{\sphinxstylestrong{Hierarchical identifier}}] \leavevmode
If the data has a hierarchical structure, e.g., individuals
in households, the variable that defines this hierarchy needs to be selected as
hierarchical identifier (see also the Section \sphinxtitleref{Risk}). This could be for instance a household ID. The hierarchical
identifier needs to be a unique ID in the complete dataset and the same for each
member of the hierarchical unit (household). If the unique hierachical indentifier is
composed of several variabels, e.g., a geographical identifier, such as region, and
a household ID which is unique within regions but not across, a unique hierarchical
identifier needs to generated before importing the data into \sphinxstyleemphasis{sdcApp}. This can be done in
a software of choice by concatenating the different components.
The household identifier can be of type …

\end{description}

\item {} \begin{description}
\item[{\sphinxstylestrong{PRAM}}] \leavevmode
If some variables are considered for application of the PRAM method (see \sphinxhref{anon.html\#PRAM}{PRAM}),
they need to be specified at this stage. PRAM variables can be of type.

\end{description}

\item {} \begin{description}
\item[{\sphinxstylestrong{Delete}}] \leavevmode
Variables that need to be deleted from the dataset for release, such as
direct identifiers, need to be selected here. Variables to be deleted can be of any type.

\end{description}

\item {} \begin{description}
\item[{\sphinxstylestrong{Number of levels}}] \leavevmode
This column shows the number of unique values in each variable. For instance a gender
variable has typically two different levels. Note that if a variable contains missing
values, this is also considered as a distinct value.

\end{description}

\item {} \begin{description}
\item[{\sphinxstylestrong{Number of missing}}] \leavevmode
This column indicates the number of missing values in each variable.
If values were set to NA, the missing value code in R, these are counted here. Other
missing value codes, such as 9, 99, 998 need to be set to NA
(see also the Section \sphinxhref{loadprepdata.html}{Set missing values to NA}).

\end{description}

\end{enumerate}

\begin{sphinxadmonition}{note}{Note:}
All variables need to be of the appropriate variable type. If the variable type of a
variable is not suitable for the selected variable function, a popup window with an error
message will appear. If necessary, the variable type needs to be changed before setting up the SDC
problem.
\end{sphinxadmonition}

\begin{figure}[htbp]
\centering
\capstart

\noindent\sphinxincludegraphics{{setupTable}.png}
\caption{Table on Anonymize tab for variable selection}\label{\detokenize{setup:fig11}}\label{\detokenize{setup:id1}}\end{figure}


\begin{savenotes}\sphinxattablestart
\centering
\sphinxcapstartof{table}
\sphinxcaption{Columns in setup table}\label{\detokenize{setup:tabsetup1}}\label{\detokenize{setup:id2}}
\sphinxaftercaption
\begin{tabulary}{\linewidth}[t]{|T|T|}
\hline
\sphinxstyletheadfamily 
Column header
&\sphinxstyletheadfamily 
Description
\\
\hline
Variable name
&
Name of variable in original dataset
\\
\hline
Type
&
Variable type in R (factor, integer, numeric, character)
\\
\hline
Key variables
&
Radio buttons to select variable as cat. or cont. key variable
\\
\hline
Weight
&
Column to select variable as weight variable
\\
\hline
Hierarchical identifier
&
Column to select variable as hierarchical identifier
\\
\hline
PRAM
&
Column to select variable for PRAM method
\\
\hline
Delete
&
Column to select variable to be deleted from released dataset
\\
\hline
Number of levels
&
Number of different values (including NA/missing) in a categorical (type factor) variable
\\
\hline
Number of missing
&
Number of records with missing value for this particular variable
\\
\hline
\end{tabulary}
\par
\sphinxattableend\end{savenotes}

Once a valid variable selection is made, a blue button will appear at the bottom of the
setup table:

\begin{figure}[htbp]
\centering
\capstart

\noindent\sphinxincludegraphics{{setupButton}.png}
\caption{Blue setup button appears below the setup table if the variable selection is valid}\label{\detokenize{setup:fig12}}\label{\detokenize{setup:id3}}\end{figure}

If a variable selection is invalid, the setup button will disappear and only reappears once
all invalid choices are corrected. Popup windows as shown in \hyperref[\detokenize{setup:fig13}]{Fig.\@ \ref{\detokenize{setup:fig13}}},
will guide the user through the variables
that need to be fixed. The most common invalid choices are the selection of more
than one function for a variable and the selection of a function that does not correspond
with the variable type.

Before clicking the blue button to setup the SDC problem, several parameters have to be set,
as outlined in the next section.

\begin{sphinxadmonition}{note}{Note:}
If an invalid variable choice is made, such as an invalid variable type
or a variable is selected for more than one choice, a pop-up window with an informative
error message is shown. An example is shown in \hyperref[\detokenize{setup:fig13}]{Fig.\@ \ref{\detokenize{setup:fig13}}}. The error
message can be closed by clicking \sphinxstyleemphasis{Continue}.
It is important to undo the invalid selection after clicking away
the error message, as this doesn’t happen automatically.
Not correcting the selection will
make it later difficult to trace back the invalid selections.
The blue setup button disappears and reappears once the problem is fixed.
\end{sphinxadmonition}

\begin{figure}[htbp]
\centering
\capstart

\noindent\sphinxincludegraphics{{setupErrorMessage}.png}
\caption{Example of a popup window with an error message after an invalid variable choice}\label{\detokenize{setup:fig13}}\label{\detokenize{setup:id4}}\end{figure}


\section{Settings}
\label{\detokenize{setup:settings}}
Besides the variable selection, there are two more parameters to be set before creating
the SDC problem: alpha and seed. Both parameters can be set with sliders
in the right sidepanel (see \hyperref[\detokenize{setup:fig14}]{Fig.\@ \ref{\detokenize{setup:fig14}}}).

\begin{figure}[htbp]
\centering
\capstart

\noindent\sphinxincludegraphics{{setupAdditionalParameters}.png}
\caption{Sliders to set additional parameters for the SDC problem}\label{\detokenize{setup:fig14}}\label{\detokenize{setup:id5}}\end{figure}


\subsection{Alpha}
\label{\detokenize{setup:alpha}}
The parameter alpha is used to compute the frequencies of keys, which is used to compute risk
measures for categorical key variables. Alpha is the weight with which a key that coincides
based on a missing value (NA) contributes to these frequencies. The default value of the
parameter alpha is 1, which means that two records that have the same key (combination
of values in key variables), are considered to coincide completely.


\subsection{Seed}
\label{\detokenize{setup:seed}}
Every time a probabilistic method is used, a different outcome is generated. For these
methods it is often recommended that a seed be set for the random number generator
if you want to produce replicable results. The seed is used to initialize the
random number generator used for probabilistic methods. In \sphinxstyleemphasis{sdcApp}, the seed can
be set to any integer value from 0 to 500. To select a value, you can click with
the mouse pointer on the slider and use the arrow keys (left and right or up and down)
to select an exact value. In \hyperref[\detokenize{setup:fig14}]{Fig.\@ \ref{\detokenize{setup:fig14}}} the seed is set at 388.

\begin{sphinxadmonition}{note}{Note:}
In order to replicate exact results when using probabilistic methods, the order in
which the methods are carried out influences the results. Therefore, besides the seed,
also the order of the operations needs to be the same. The seed changes when used in
the random number generator. When the undo button is used (see ), the seed is not
reset to the value prior to the reverted step.
\end{sphinxadmonition}


\section{Summary view}
\label{\detokenize{setup:summary-view}}
After setting up the SDC problem, the application jumps automatically to the summary
view of the \sphinxstyleemphasis{Anonymize} tab. When an SDC problem is available, the \sphinxstyleemphasis{Anonymize} tab
provides a summary of the SDC problem and allows to apply anonymization methods.

This tab first shows a Summary overview of the problem. The content of the summary page varies with
the SDC problem. For example, if no numerical key variables were selected, the information on
numeric key variables is omitted. Fig shows the summary page.


\chapter{Risk measurement}
\label{\detokenize{risk::doc}}\label{\detokenize{risk:risk-measurement}}

\section{Summary view}
\label{\detokenize{risk:summary-view}}

\subsection{Global risk measure}
\label{\detokenize{risk:global-risk-measure}}
For categorical variables


\subsection{\protect\(k\protect\)-anonymity}
\label{\detokenize{risk:anonymity}}
The risk measure \(k\)-anonymity is based on the principle that, in a safe
dataset, the number of individuals sharing the same combination of
values (keys) of categorical quasi-identifiers should be higher than a
specified threshold \(k\). \(k\)-anonymity is a risk
measure based on the microdata to be released, since it only takes the
sample into account. An individual violates \(k\)-anonymity if the
sample frequency count \(f_{k}\) for the key \(k\) is smaller
than the specified threshold \(k\). For example, if an
individual has the same combination of quasi-identifiers as two other
individuals in the sample, these individuals satisfy 3-anonymity but
violate 4-anonymity. In the dataset, six individuals
satisfy 2-anonymity and four violate 2-anonymity. The individuals that
violate 2-anonymity are sample uniques. The risk measure is the number
of observations that violates k-anonymity for a certain value of \sphinxstyleemphasis{k},
which is
\begin{equation*}
\begin{split}\sum_{i}^{}{I(f_{k} < k)},\end{split}
\end{equation*}
where \(I\) is the indicator function and \(i\) refers to the
\(i\)$^{\text{th}}$ record. This is simply a count of the number of
individuals with a sample frequency of their key lower than \(k\).
The count is higher for larger \(k\), since if a record satisfies
\(k\)-anonimity, it also satisfies \((k + 1)\)-anonimity. The
risk measure \(k\)-anonymity does not consider the sample weights,
but it is important to consider the sample weights when determining the
required level of \(k\)-anonymity. If the sample weights are large,
one individual in the dataset represents more individuals in the target
population, the probability of a correct match is smaller, and hence the
required threshold can be lower. Large sample weights go together with
smaller datasets. In a smaller dataset, the probability to find another
record with the same key is smaller than in a larger dataset. This
probability is related to the number of records in the population with a
particular key through the sample weights.

In the summary view

\begin{figure}[htbp]
\centering
\capstart

\noindent\sphinxincludegraphics{{summary_k_anon}.png}
\caption{Information on \(k\)-anonymity violators in summary view}\label{\detokenize{risk:fig71}}\label{\detokenize{risk:id2}}\end{figure}


\subsection{Risk measures for numerical key variables}
\label{\detokenize{risk:risk-measures-for-numerical-key-variables}}

\subsection{Household risk}
\label{\detokenize{risk:household-risk}}
If household identifier is selected, household risk will automatically be displayed.


\section{Detailed view}
\label{\detokenize{risk:detailed-view}}
The Risk/Utility tab provides more detailed information on risk measures and records at
(high) risk.


\subsection{Risky observations}
\label{\detokenize{risk:risky-observations}}

\subsection{SUDA}
\label{\detokenize{risk:suda}}
The SUDA algorithm identifies all the MSUs in the sample, which in turn
are used to assign a SUDA score to each record. This score indicates how
“risky” a record is. The potential risk of the records is determined
based on two observations:
\begin{itemize}
\item {} 
The smaller the size of the MSU within a record (i.e., the fewer
variables are needed to reach uniqueness), the greater the risk of
the record

\item {} 
The larger the number of MSUs possessed by a record, the greater the
risk of the record

\end{itemize}

\begin{figure}[htbp]
\centering
\capstart

\noindent\sphinxincludegraphics{{risk_suda_setup}.png}
\caption{Compute SUDA scores}\label{\detokenize{risk:id1}}\label{\detokenize{risk:id3}}\end{figure}

\begin{figure}[htbp]
\centering
\capstart

\noindent\sphinxincludegraphics{{risk_suda_result}.png}
\caption{Result of SUDA calculation}\label{\detokenize{risk:fig72}}\label{\detokenize{risk:id4}}\end{figure}


\subsection{l-diversity}
\label{\detokenize{risk:l-diversity}}
A dataset
satisfies \(l\)-diversity if for every key \(k\) there are at least
\(l\) different values for each of the sensitive variables. In the
example, the first two individuals satisfy only 1-diversity, individuals
4 and 6 satisfy 2-diversity. The required level of \(l\)-diversity
depends on the number of possible values the sensitive variable can
take. If the sensitive variable is a binary variable, the highest level
if \(l\)-diversity that can be achieved is 2. A sample unique will
always only satisfy 1-diversity.

To compute \(l\)-diversity for sensitive variables in sdcApp


\subsection{k-anonymity}
\label{\detokenize{risk:k-anonymity}}
\begin{figure}[htbp]
\centering
\capstart

\noindent\sphinxincludegraphics{{summary_k_anon}.png}
\caption{Information on \(k\)-anonymity violators for any level of \(k\)}\label{\detokenize{risk:fig73}}\label{\detokenize{risk:id5}}\end{figure}


\chapter{Anonymization methods}
\label{\detokenize{anon::doc}}\label{\detokenize{anon:anonymization-methods}}
Once the disclosure risk is evaluated and is too high for release, SDC methods need
to be applied to the variables to reduce the risk. This process is a iterative, i.e.,
after applying a certain method with a set of parameters, the disclosure risk
needs to be reassessed and the information loss needs to be evaluated. If the result is not
satisfactory, other methods can be applied to other variables. It is also possible to
undo the method and reapply the same method with a different  set of parameters.

In this section, we provide a brief description of common SDC methods for microdata and
show how to use these in \sphinxstyleemphasis{sdcApp}. For more information on the choice of the
appropriate method and more detailed information on the methods themselves, we refer to …


\section{Recoding}
\label{\detokenize{anon:recoding}}

\subsection{Global recoding}
\label{\detokenize{anon:global-recoding}}
Global recoding combines several categories of a categorical variable or constructs
intervals for continuous variables. This reduces the number of categories available
in the data and potentially the disclosure risk, especially for categories with few
observations, but also, importantly, it reduces the level of detail of information
available to the analyst.

Variable needs to be of type factor and key variable

Variables can already be recoded before on data tab

\begin{figure}[htbp]
\centering
\capstart

\noindent\sphinxincludegraphics{{anonGlobalRecodeSettings}.png}
\caption{Settings for global recoding to recode the variable age}\label{\detokenize{anon:fig81}}\label{\detokenize{anon:id1}}\end{figure}


\subsection{Top and bottom coding}
\label{\detokenize{anon:top-and-bottom-coding}}
Top and bottom coding are similar to global recoding, but instead of recoding all values,
only the top and/or bottom values of the distribution or categories are recoded. This can
be applied only to ordinal categorical variables and (semi-)continuous variables, since
the values have to be at least ordered. Top and bottom coding is especially useful if
the bulk of the values lies in the center of the distribution with the peripheral
categories having only few observations (outliers). Examples are age and income; for
these variables, there will often be only a few observations above certain thresholds,
typically at the tails of the distribution. The fewer the observations within a category,
the higher the identification risk. One solution could be grouping the values at the tails
of the distribution into one category. This reduces the risk for those observations, and,
importantly, does so without reducing the data utility for the other observations in the
distribution.

\begin{figure}[htbp]
\centering
\capstart

\noindent\sphinxincludegraphics{{anonTopcodingSettings}.png}
\caption{Settings for topcoding the variable income at 8 million}\label{\detokenize{anon:fig82}}\label{\detokenize{anon:id2}}\end{figure}

\begin{sphinxadmonition}{note}{Note:}
Top and bottom coding ca only be applied to numeric variables. If age, as in our example,
is converted to factor, the global recoding method needs to be used, in order to
topcode age by grouping all values above the threshold.
\end{sphinxadmonition}

\begin{sphinxadmonition}{note}{Note:}
IT is advised to use a replacement value different than the threshold value,
such as the weighted mean or median to reduce information loss. The replacement
value needs to be computed in a different software and manually inserted in \sphinxstyleemphasis{sdcApp}.
\end{sphinxadmonition}


\section{k-Anonimity / local suppression}
\label{\detokenize{anon:k-anonimity-local-suppression}}
It is common in surveys to encounter values for certain variables or combinations
of quasi-identifiers (keys) that are shared by very few individuals. When this occurs,
the risk of re-identification for those respondents is higher than the rest of the
respondents (see the Section k-anonymity). Often local suppression is used after
reducing the number of keys in the data by recoding the appropriate variables.
Recoding reduces the number of necessary suppressions as well as the computation
time needed for suppression. Suppression of values means that values of a variable
are replaced by a missing value (NA in R). The the Section k-anonymity discusses how
missing values influence frequency counts and k-anonymity.

\begin{figure}[htbp]
\centering
\capstart

\noindent\sphinxincludegraphics{{anonLocSupSettings}.png}
\caption{Settings for local suppression to achieve 3-anonimity}\label{\detokenize{anon:fig83}}\label{\detokenize{anon:id3}}\end{figure}


\subsection{Importance}
\label{\detokenize{anon:importance}}
\begin{figure}[htbp]
\centering
\capstart

\noindent\sphinxincludegraphics{{anonLocSupSettingsImportance}.png}
\caption{Importance settings for local suppression}\label{\detokenize{anon:fig84}}\label{\detokenize{anon:id4}}\end{figure}


\subsection{Subsets}
\label{\detokenize{anon:subsets}}
\begin{figure}[htbp]
\centering
\capstart

\noindent\sphinxincludegraphics{{anonLocSupSettingsSubset}.png}
\caption{Subset settings for local suppression}\label{\detokenize{anon:fig85}}\label{\detokenize{anon:id5}}\end{figure}


\bigskip\hrule\bigskip



\section{PRAM}
\label{\detokenize{anon:pram}}
\begin{figure}[htbp]
\centering
\capstart

\noindent\sphinxincludegraphics{{anonPRAMsettings}.png}
\caption{Settings for PRAM}\label{\detokenize{anon:fig86}}\label{\detokenize{anon:id6}}\end{figure}

\begin{figure}[htbp]
\centering
\capstart

\noindent\sphinxincludegraphics{{anonPRAMsettingsMatrix}.png}
\caption{Settings for PRAM with customized transition matrix}\label{\detokenize{anon:fig87}}\label{\detokenize{anon:id7}}\end{figure}


\section{Suppress values with high risk}
\label{\detokenize{anon:suppress-values-with-high-risk}}
\begin{figure}[htbp]
\centering
\capstart

\noindent\sphinxincludegraphics{{anonSuppressSettings}.png}
\caption{Settings for suppressing values in records with high risk}\label{\detokenize{anon:fig88}}\label{\detokenize{anon:id8}}\end{figure}


\section{Top/Bottom coding}
\label{\detokenize{anon:top-bottom-coding}}

\section{Microaggregation}
\label{\detokenize{anon:microaggregation}}
\begin{figure}[htbp]
\centering
\capstart

\noindent\sphinxincludegraphics{{anonMicroaggregationSettingsCluster}.png}
\caption{Settings for microaggregation}\label{\detokenize{anon:fig89}}\label{\detokenize{anon:id9}}\end{figure}

\begin{figure}[htbp]
\centering
\capstart

\noindent\sphinxincludegraphics{{anonMicroaggregationSettingsAdditional}.png}
\caption{Additional settings for microaggregation}\label{\detokenize{anon:fig810}}\label{\detokenize{anon:id10}}\end{figure}

\begin{figure}[htbp]
\centering
\capstart

\noindent\sphinxincludegraphics{{anonMicroaggregationSettingsCluster}.png}
\caption{Cluster settings for microaggregation}\label{\detokenize{anon:fig811}}\label{\detokenize{anon:id11}}\end{figure}


\section{Adding noise}
\label{\detokenize{anon:adding-noise}}

\section{Rank swapping}
\label{\detokenize{anon:rank-swapping}}

\section{Undo}
\label{\detokenize{anon:undo}}
Finding an anonymization strategy for a microdata dataset is a trial-and-error process.
The effect on risk and utility of different methods with different parameter settings can
only be assessed by executing the methods on the actual dataset. Therefore, it is unlikely
to find a satisfactory anonymization strategy at the first attempt. Before another method
is applied, the previous method needs to be canceled. In \sphinxstyleemphasis{sdcApp} it is possible
to undo the last method applied with one click. To test the effect of a combination
of several methods, which is recommended, it is necessary to cancel several steps.
To do so, the state of the SDC problem before applying the methods is saved to disk and can
be reloaded afterwards. This is the same as canceling several steps. Both methods are
described below.


\subsection{Undo one step}
\label{\detokenize{anon:undo-one-step}}
In order to undo one step,

risk measures etc are also reset, script not, random seed not


\subsection{Undo several steps}
\label{\detokenize{anon:undo-several-steps}}
Recommended to

Save and reload


\chapter{Utility measurement}
\label{\detokenize{utility::doc}}\label{\detokenize{utility:utility-measurement}}

\section{General utility measures in \sphinxstyleemphasis{sdcApp}}
\label{\detokenize{utility:general-utility-measures-in-sdcapp}}

\subsection{Compare summary statistics}
\label{\detokenize{utility:compare-summary-statistics}}

\subsubsection{Categorical variables}
\label{\detokenize{utility:categorical-variables}}

\subsubsection{Continuous variables}
\label{\detokenize{utility:continuous-variables}}

\subsection{IL1s measure}
\label{\detokenize{utility:il1s-measure}}

\section{Customized utility measures}
\label{\detokenize{utility:customized-utility-measures}}
As the statistical analyses based on the microdata depend, amongst others,
on to the topic of the survey, the country and the definition of the variables,
it is not feasible to include all these measures in \sphinxstyleemphasis{sdcApp}. Instead, it is recommended
to compute the statistics and indicators and perform statistical an econometric
analyses on the original and anonymized datasets and evaluate the differences in the results.
If a publication based on the microdata is already published, it is recommended
to recompute the statistics in these publications from the anonymized dataset.

The approach is to compare the indicators calculated on the untreated data and the
data after anonymization with different methods. If the differences between the
indicators are not too large, the anonymized dataset can be released for use by
researchers. It should be taken into account that indicators calculated on samples
are estimates with a certain variance and confidence interval. Therefore, for sample
data, it is more informative to compare the overlap of confidence intervals and/or
to evaluate whether the point estimate calculated after anonymization is contained
within the confidence interval of the original estimate.

\begin{sphinxadmonition}{note}{Note:}
Some analyses may no longer be possible or not possible in exactly the same way.
E.g. regression on age if age is recoded in 5-year intervals.
\end{sphinxadmonition}

In order to do so, it is posible to export the dataset at any point in \sphinxstyleemphasis{sdcApp}.
See Export dataset. Several datasets can be exported after applying different methods
with different parameters settings to compare the information loss resulting from
the anonymization. This information can be used to select the anonymization methods
as well as to inform the user about the implications of the anonymization on
the validity of the dataset for analysis.


\chapter{Export data and reports}
\label{\detokenize{export::doc}}\label{\detokenize{export:export-data-and-reports}}

\section{Export anonymized dataset}
\label{\detokenize{export:export-anonymized-dataset}}
\sphinxstyleemphasis{sdcApp} supports datasets in several foreign data formats. The file formats that are
supported for loading microdata are also supported for export (cf. \hyperref[\detokenize{export:tab101}]{Table \ref{\detokenize{export:tab101}}}).

In order to export the file, click on  \sphinxstylestrong{Anonymized Data} in the left sidebar on the
\sphinxstylestrong{Export Data} tab. The dataset shown is the file as it will be exported. Select the
appropriate file format with the radiobuttons underneath the data.

In case the microdata is exported as csv or STATA file, additional options need to
be specified. For a csv file, whether first row should include column names,
the field separator as well as the decimal separator. For STATA files, the version of STATA
needs to be specified. STATA cannot STATA files saved for a higher version.

Option to randomize the order of the records. Order may reveal values, e.g.
ordered by region with suppressed region value Need to replace existing ID

In order to export the dataset, click on blue button \sphinxstylestrong{Save dataset}. The dataset is saved
to the
, exported two with file name… Exported to
the microdata, select


\begin{savenotes}\sphinxattablestart
\centering
\sphinxcapstartof{table}
\sphinxcaption{Data formats compatible with sdcApp}\label{\detokenize{export:tab101}}\label{\detokenize{export:id1}}
\sphinxaftercaption
\begin{tabulary}{\linewidth}[t]{|T|T|}
\hline
\sphinxstyletheadfamily 
Software
&\sphinxstyletheadfamily 
File extension
\\
\hline
R/RStudio
&
.rdata
\\
\hline
SPSS
&
.sav
\\
\hline
SAS
&
.sas7bdat
\\
\hline
CSV
&
.csv, .txt
\\
\hline
STATA
&
.dta
\\
\hline
\end{tabulary}
\par
\sphinxattableend\end{savenotes}

Also for intermediate export
\begin{description}
\item[{..NOTE::}] \leavevmode
Categorical variables (type factor in \sphinxstyleemphasis{sdcApp}) that were had a value and a
label in the input dataset are

labels (variable, value), coding 0,1 to 1,2 etc.

\end{description}

Extra for STATA input files: change variable labels for STATA files


\section{Exporting reports}
\label{\detokenize{export:exporting-reports}}
It is extremely important to document the SDC process of microdata both for internal
use as well as for external use. The internal report should contain detailed descriptions
of all steps carried out as well as reasoning for


\subsection{Internal report}
\label{\detokenize{export:internal-report}}
An important step in the SDC process is reporting, both internal and external.
Internal reporting contains the exact description of anonymization methods used,
parameters but also the risk measures before and after anonymization. This allows
replication of the anonymized dataset and is important for supervisory authorities/bodies
to ensure the anonymization process is sufficient to guarantee anonymity according
to the applicable legislation.

Report is just technical overview, not complete

file path, name of file

\begin{figure}[htbp]
\centering
\capstart

\noindent\sphinxincludegraphics{{reproScript}.png}
\caption{Exporting an internal report}\label{\detokenize{export:fig102}}\label{\detokenize{export:id2}}\end{figure}


\subsection{External report}
\label{\detokenize{export:external-report}}
The external report

External reporting informs the user that the data has been anonymized,
provides information for valid analysis on the data and explains the limitations to
the data as a result of the anonymization. A brief description of the methods used
can be included. The release of anonymized microdata should be accompanied by the
usual metadata of the survey (survey weight, strata, survey methodology) as well as
information on the anonymization methods that allow researchers to do valid analysis
(e.g., amount of noise added, transition matrix for PRAM).

file path, name of file


\chapter{Reproducibility}
\label{\detokenize{reproducibility::doc}}\label{\detokenize{reproducibility:reproducibility}}
Reproducibility is key to the SDC process, as …


\section{Exporting \sphinxstyleemphasis{R} script}
\label{\detokenize{reproducibility:exporting-r-script}}
\sphinxstyleemphasis{sdcApp} is a GUI for the \sphinxstyleemphasis{R} package \sphinxstyleemphasis{sdcMicro}. All steps executed in \sphinxstyleemphasis{sdcApp} are translated
into \sphinxstyleemphasis{R} commands. Therefore, the full anonymization can also be performed from command-line
in \sphinxstyleemphasis{R}. While carrying out the anonymization process, the code to perform the same action
from command-line is generated. The code can be viewed and exported on the \sphinxstylestrong{Reproducibility}
tab by selecting \sphinxstylestrong{View the current script} from the left sidebar (cf. \hyperref[\detokenize{reproducibility:fig111}]{Fig.\@ \ref{\detokenize{reproducibility:fig111}}}).
The script also contains comments, which are the lines starting with the hash tag (\sphinxcode{\sphinxupquote{\#}}).
These comments are meant to help with the interpretation of the code blocks.

\begin{figure}[htbp]
\centering
\capstart

\noindent\sphinxincludegraphics{{reproScript}.png}
\caption{\sphinxstyleemphasis{R} script to reproduce anonymization process after setting up SDC problem}\label{\detokenize{reproducibility:fig111}}\label{\detokenize{reproducibility:id5}}\end{figure}

The goal of the \sphinxstyleemphasis{R} script is threefold:
1) To reproduce the steps taken in the anonymization process. This guarantees
the reproducibility, since the all variabele selections and parameters are contained in the
code and the order of the application of different methods is preserved in the code.
2) As a starting point to learn \sphinxstyleemphasis{R} and use \sphinxstyleemphasis{sdcMicro} from \sphinxstyleemphasis{R} command-line. Especially for
users with some degree of familiarity with \sphinxstyleemphasis{R}, the script
3) To rerun the same methods with different parameter settings without the need to make
all selections by mouseclick in the GUI. It’s relatively easy to change the parameter settings
in the \sphinxstyleemphasis{R} code and rerun the code. However, the code does not include commands to shw
the results.

By clicking on the blue button \sphinxstylestrong{Save script to file} at the top of the page, the script
is saved as \sphinxstyleemphasis{R} script (extension .R) on disk to the selected storage path on the
\sphinxstylestrong{About/Help} tab (see \sphinxhref{introsdcApp.html}{Introduction to sdcApp}).
The filename of the exported script starts with ‘exportedScript\_sdcMicro’ followed
by a date and time stamp, e.g., ‘exportedScript\_sdcMicro\_20181010\_1212.R’.

In order to run the script in \sphinxstyleemphasis{R}, open the saved script in \sphinxstyleemphasis{RStudio}. The only
thing to do is to change the path of the input file to the actual file path on your computer.
In the second line of the \sphinxstyleemphasis{R} script,

\begin{sphinxadmonition}{note}{Note:}
In case a method was applied in \sphinxstyleemphasis{sdcApp} and subsequently reverted by using the \sphinxstylestrong{Undo}
button, the method is not erased from the script, but rather the undo command is added.
For example, if local suppression was applied and reverted, this appears as follows in
the script:

\fvset{hllines={, ,}}%
\begin{sphinxVerbatim}[commandchars=\\\{\},numbers=left,firstnumber=1,stepnumber=1]
\PYG{c+c1}{\PYGZsh{}\PYGZsh{} Local suppression to obtain k\PYGZhy{}anonymity}
sdcObj \PYG{o}{\PYGZlt{}\PYGZhy{}} kAnon\PYG{p}{(}sdcObj\PYG{p}{,} importance\PYG{o}{=}\PYG{k+kt}{c}\PYG{p}{(}\PYG{l+m}{1}\PYG{p}{,}\PYG{l+m}{2}\PYG{p}{,}\PYG{l+m}{3}\PYG{p}{)}\PYG{p}{,} combs\PYG{o}{=}\PYG{k+kc}{NULL}\PYG{p}{,} k\PYG{o}{=}\PYG{k+kt}{c}\PYG{p}{(}\PYG{l+m}{3}\PYG{p}{)}\PYG{p}{)}
sdcObj \PYG{o}{\PYGZlt{}\PYGZhy{}} undolast\PYG{p}{(}sdcObj\PYG{p}{)}
\end{sphinxVerbatim}

This code preceding the :code:{\color{red}\bfseries{}{}`}undoLast{}`command and the :code:{\color{red}\bfseries{}{}`}undoLast{}`command
can be deleted without changing the results.
\end{sphinxadmonition}


\section{Exporting reports}
\label{\detokenize{reproducibility:exporting-reports}}
It is extremely important to document the SDC process of microdata both for internal
use as well as for external use. The internal report should contain detailed descriptions
of all steps carried out as well as reasoning for


\subsection{Internal report}
\label{\detokenize{reproducibility:internal-report}}
An important step in the SDC process is reporting, both internal and external.
Internal reporting contains the exact description of anonymization methods used,
parameters but also the risk measures before and after anonymization. This allows
replication of the anonymized dataset and is important for supervisory authorities/bodies
to ensure the anonymization process is sufficient to guarantee anonymity according
to the applicable legislation.

Report is just technical overview, not complete

\begin{figure}[htbp]
\centering
\capstart

\noindent\sphinxincludegraphics{{reproScript}.png}
\caption{Exporting an internal report}\label{\detokenize{reproducibility:fig112}}\label{\detokenize{reproducibility:id6}}\end{figure}


\subsection{External report}
\label{\detokenize{reproducibility:external-report}}
The external report

External reporting informs the user that the data has been anonymized,
provides information for valid analysis on the data and explains the limitations to
the data as a result of the anonymization. A brief description of the methods used
can be included. The release of anonymized microdata should be accompanied by the
usual metadata of the survey (survey weight, strata, survey methodology) as well as
information on the anonymization methods that allow researchers to do valid analysis
(e.g., amount of noise added, transition matrix for PRAM).


\chapter{Undo}
\label{\detokenize{undo::doc}}\label{\detokenize{undo:undo}}
Microdata anonymization is a trial-and-error process. It is necessary to several
methods, each with different parameter settings, to find optimal set of
anonymization measures that minimize the information loss, while reducing the risk of
disclosure to an acceptable level. Before applying an alternative method to the same
variable or set of variables, it is important to undo the previously applied methods.
Only in this way, it is possible to compare the effect on risk and information
loss of a particular method or parameter setting. For instance to compare the
effect of recoding the age variable in 5 or 10 year intervals, it is necessary to first
undo the recoding in 5-year intervals before recoding in 10 year intervals.

In \sphinxstyleemphasis{sdcApp} it is possible to undo the last anonymization step. In order to undo several
steps, it is required to save and reload the SDC problem. Both ways are explained below.


\section{Undo one step}
\label{\detokenize{undo:undo-one-step}}
In order to undo one step, go to the \sphinxstylestrong{Undo} tab.
The screen shows

risk measures etc are also reset, script not (completely), random seed not


\section{Undo several steps}
\label{\detokenize{undo:undo-several-steps}}
Save and reload

Recommended to save the SDC problem after each method to be able to reload. This is also
practical if \sphinxstyleemphasis{sdcApp} or \sphinxstyleemphasis{R} crash. Also useful to continue working at a later point of
transfer a problem to a different computer


\subsection{Save a previously saved problem}
\label{\detokenize{undo:save-a-previously-saved-problem}}

\subsection{Load a previously saved problem}
\label{\detokenize{undo:load-a-previously-saved-problem}}\begin{description}
\item[{..NOTE::}] \leavevmode
Different file names for different files (data, sdcProblem)

\end{description}



\renewcommand{\indexname}{Index}
\printindex
\end{document}